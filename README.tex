% ============================================================================
%  Copyright (c) 2015 IEEE@UCR
%  Description: This is the LaTeX source for the documentation on ...
%    Member_SignIn documentation.
% ============================================================================
\documentclass[12pt]{article}
\usepackage[USenglish]{babel}
\usepackage{parskip}
\usepackage{fullpage}
\usepackage{keystroke}
\usepackage{hyperref}
\usepackage{float}

\floatstyle{boxed}
\restylefloat{figure}

\hypersetup{%
	colorlinks=false,%
	pdfborderstyle={/S/U/W 2}%
}

\begin{document}

\title{IEEE@UCR Member\_SignIn}
\author{Brandon Lu}
\maketitle

This document is written in order to provide some form of documentation for
the IEEE@UCR Member\_SignIn system.  Although design rationale will be
covered, please note that that is not the main purpose of this document,
and that further details about implementation and a large amount of jokes
can be ``bought over'' by treating the author to a meal.  He may also follow
you home, so please be wise.

Please note that this document is written in a ``backwards fashion'' such that
the installation and configuration are last in the document, and the end-user
usage is first.
This is done with the 
logic that it may be read from the most-used portion to the least used
portion.

Please also note that this document is probably going to be under active
construction for a long, long time.  The author does not plan on abandoning
software development for some time, but he is in a big time-crunch with
various activities.
\newpage

\section{Operation}
This section primarily details how to use the system.  If everything is set up
correctly, it's pretty simple.  Troubleshooting any system will be the most
complicated part of anyone's job.

\subsection{Primary Authentication}
\begin{figure}[h]
{\tt IEEE@UCR Card Login System\\
Please swipe your card or press enter for manual entry.\\
> }
\caption{Primary authentication mode.}
\label{fig:pauth}
\end{figure}
Primary authentication can be completed in one of two ways.  An example of
the screen that will be seen in this first mode is seen at Figure
\ref{fig:pauth}.
\subsubsection{Card Swipe}
Card swipe authentication is achieved by reading the input from stdin provided
by a standard card reader on the student's ID.  Be sure your keyboard is set
to ``{\tt US},'' this has become a problem in the past with administrators who have
their keyboards set to something other than ``{\tt US}.''

The specifics can be found in the code if necessary, but it should not be
necessary.
\subsubsection{Student ID Entry}
\begin{figure}[h]
{\tt IEEE@UCR Card Login System\\
Manual SID entry mode activated.\\
UCR SID: 
}
\caption{Student ID Entry mode}
\label{fig:sidauth}
\end{figure}
Student ID entry is activated upon pressing the ``enter'' \Return key on an 
empty input.  That is, no characters are entered before pressing the ``enter''
\Return key.

The terminal will prompt for the user's student id, and only accepts an id
with ``{\tt860\textbackslash d\{6\}}"\footnote{You can read more about PERL
regular
 expressions (albeit, this C program does not use PERL) from PERL's online
 documentation. {\tt\url{http://perldoc.perl.org/perlre.html}}
} or ``{\tt861\textbackslash d\{6\}}."  When the user inputs an incorrect ID,
the user will be redirected to the primary authentication screen.

An example of the SID entry screen is in Figure \ref{fig:sidauth}.
\subsection{User Information}
The Member\_SignIn system also likes to ask the user for information.  This is
necessary to send said user coherent emails.
\subsubsection{Last Name}
\begin{figure}[h]
{\tt IEEE@UCR Card Login System\\
Manual Last Name entry mode activated.\\
Last Name:}
\caption{Last Name Entry Mode}
\label{fig:lname}
\end{figure}
Usually, the user's last name is retrieved from a card swipe, however; this is
not the case with a direct student ID sign in.  A database lookup will also
reveal the user's last name.

Upon entering a last name that contains non-alpha characters, the user will be
forced to keep entering a last name until it finally makes sense.
\footnote{This is probably not the best way to deal with easily frustrated
or computer-challenged users.}

An example of this mode can be seen in Figure \ref{fig:lname}.
\subsubsection{First Name}
\begin{figure}[h]
{\tt IEEE@UCR Card Login System\\
Manual First Name entry mode activated.\\
First Name:
}
\caption{First Name Entry Mode}
\label{fig:fname}
\end{figure}
Usually, the user's first name is retrieved from a card swipe, however; this is
not the case with a direct student ID sign in.  A database lookup will also
reveal the user's first name.

Upon entering a first name that contains non-alpha characters, the user will be
forced to keep entering a first name until it finally makes sense.

An example of this mode can be seen in Figure \ref{fig:fname}.
\subsubsection{Email}
\begin{figure}[h]
{\tt IEEE@UCR Card Login System\\
Manual email entry mode activated.\\
UCR email:
}
\caption{Email Entry Mode}
\label{fig:email}
\end{figure}
A database lookup will usually reveal the user's email, but this is not the
case when the user does not have an existing record.

These emails specifically are limited to ucr.edu emails.  The PERL
verification string \footnote{This program has no PERL, but PERL is the
text processing language of choice.} should be 
``{\tt[:alphanum:]+@[alpha]*\textbackslash .?ucr\textbackslash .edu}.''

An example of the email entry mode can be seen in Figure \ref{fig:email}.
\subsubsection{Member Number}
A database lookup should reveal the user's member number, but this is not
always the case.

The user shall be forever bothered with this unless the user
already has a member number filed with the IEEE@UCR.  This is also a useful
feature for cross-referencing with official IEEE records.
\subsection{Verification}
During this step, the user is able to change any of his or her information by
entering the bolded characters onto the terminal
\footnote{There is also an undocumented administrative mode, but this is not
to be revealed in this document.  It is noted that one must change a
character flag within the user informaiton storage database in order to gain
access to the administrative modes in the first place.}.

Be aware that by selecting any field, the user will effectively delete their
information from that field before providing new information.
There is currently no way to revert the information back to what it was
before, and there are no plans to do so at the moment.

\section{Configuration}
The program has variables.  Go change them if something looks off.

\subsection{Organizaiton Informaiton}

\subsection{Database Information}

\section{Installation}
By this point, if this is not extraneous information, C is not an interpreted
language.
\subsection{Database Configuration}

\subsection{Program Configuration}

\subsection{Dependencies}

\subsection{Compilation}

\section{Troubleshooting}
For now, your best hope is to contact your local system administrator.
Assuming that this software is not widely distributed, just call the 
author over.

In the worst-case scenario, it is acceptable to enter everything into a large
text file that will be manually interpreted at the end of the meeting.

\end{document}

